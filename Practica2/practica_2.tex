\part{Práctica 2}
\section{Ejercicio 1}
\begin{center}
    \parbox{12cm}{\justify\textit{Explica brevemente qué es un sistema de control de versiones distribuido y sus diferencias con respecto a uno centralizado. Entra en la web de Git y revisa la documentación. Busca manuales, tutoriales, etc. de Git. Describe y compara con Git algún otro sistema de control de versiones distribuido (Mercurial, etc.).
    }}
\end{center}




\section{Ejercicio 2}
\begin{center}
    \parbox{12cm}{\justify\textit{
        Establece tu nombre de usuario y dirección de correo electrónico. Esto es importante porque las confirmaciones de cambios (commits) en Git usan esta información, y es introducida de manera inmutable en los commits que envías. Utiliza tu usuario y correo de la uco, por ejemplo:}}
    \end{center}


\begin{lstlisting}[xleftmargin=.16\textwidth]
$ git config --global user.name "i22lojal"
$ git config --global user.email "i22lojal@uco.es"
\end{lstlisting}


\section{Ejercicio 3}
\begin{center}
    \parbox{12cm}{\justify\textit{Vamos a crear un nuevo repositorio desde cero (puedes usar uno tuyo o seguir el breve ejemplo que se describe a continuación). Crea en tu cuenta un nuevo directorio que se llame “reposlycs”. Entra en dicho directorio e inicializa el repositorio (\code{\$git init}). Comprueba la creación del directorio .git. ¿Qué función tiene este directorio?
    }}
\end{center}


\section{Ejercicio 4}
\begin{center}
    \parbox{12cm}{\justify\textit{Crea un directorio que cuelgue del anterior llamado ``include''.
    }}
\end{center}

\section{Ejercicio 5}
\begin{center}
    \parbox{12cm}{\justify\textit{
        En el directorio ``reposlycs'' escribe un fichero llamado “helloworld.c” que contenga el siguiente código:
    }}
\end{center}

\begin{lstlisting}[style=CStyle,xleftmargin=.3\textwidth]
#include <stdio.h>
int main(void)
{
    printf("Hello World!"); 
}\end{lstlisting}
    
\begin{center}
    \parbox{12cm}{\justify\textit{
        y un fichero ``include/myinclude.h'' que contenga:
    }}
\end{center}

\begin{lstlisting}[style=CStyle,xleftmargin=.3\textwidth]
include <stdio.h>
void f()
{
    printf("Hello World \n");
}\end{lstlisting}

\section{Ejercicio 6}
\begin{center}
    \parbox{12cm}{\justify\textit{
        Añade estos ficheros a tu repositorio y comprueba el estado del repositorio con “git status”. Añade los cambios al stage (\code{\$git add ...}). Haz el primer commit con el mensaje ``Initial commit'' y vuelve a comprobar el estado de tu repositorio.
    }}
\end{center}

\section{Ejercicio 7}
\begin{center}
    \parbox{12cm}{\justify\textit{
        Compila y comprueba lo que ocurre con el ejecutable generado. ¿Se podría añadir el ejecutable al staging y luego al commit?. Haz una prueba. ¿Tendría sentido hacer commit de un ejecutable?
    }}
\end{center}

\section{Ejercicio 8}
\begin{center}
    \parbox{12cm}{\justify\textit{
        Abre una rama que se llama ``branchA'' y modifica el fichero de la función f() con lo necesario para que f() quede así:
    }}
\begin{lstlisting}[style=CStyle,xleftmargin=.3\textwidth]
void f()
{
    char c1[100]= "Hello world";
    char c2[100]= ", I am <your name here>";
    printf("%s\n", strcat(c1, c2));
    return 0;
}\end{lstlisting}
\end{center}

\section{Ejercicio 9}
\begin{center}
    \parbox{12cm}{\justify\textit{
        Realiza el commit con los cambios mencionados en la branchA y el mensaje ``Now with strcat()''.
    }}
\end{center}

\section{Ejercicio 10}
\begin{center}
    \parbox{12cm}{\justify\textit{
        Cambia a la rama master (\code{\$git checkout master}). En la rama master añade un comentario al inicio del fichero ``helloworld.c'' con la fecha y tu nombre como autor del fichero. Realiza el commit de este cambio en la rama master. Con el mensaje ``Now with date and author''
    }}
\end{center}

\section{Ejercicio 11}
\begin{center}
    \parbox{12cm}{\justify\textit{
        Desde la rama master, fusiona la rama master con la rama branchA. En este caso no hay ningún conflicto, ambas ramas han modificado distintos archivos. Es un merge sin conflicto.
    }}
\end{center}

\section{Ejercicio 12}
\begin{center}
    \parbox{12cm}{\justify\textit{
        Repite el proceso del punto anterior pero ahora creando un conflicto (merge con conflicto).
    }}
\end{center}

\section{Ejercicio 13}
\begin{center}
    \parbox{12cm}{\justify\textit{
        Sube este proyecto a GitHub.
    }}
\end{center}

\section{Ejercicio 14}
\begin{center}
    \parbox{12cm}{\justify\textit{
        Bájate un proyecto de GitHub de uno de tus compañeros (o de cualquier otra  persona) a tu cuenta local.
    }}
\end{center}