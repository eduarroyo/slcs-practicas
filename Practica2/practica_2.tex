\part{Práctica 2}
\setcounter{section}{0}
\section{Ejercicio 1}
\begin{center}
    \parbox{12cm}{\justify\textit{Explica brevemente qué es un sistema de control de versiones distribuido y sus diferencias con respecto a uno centralizado. Entra en la web de Git y revisa la documentación. Busca manuales, tutoriales, etc. de Git. Describe y compara con Git algún otro sistema de control de versiones distribuido (Mercurial, etc.).
    }}
\end{center}

Un \textbf{sistema de control de versiones distribuido} es una forma de control de versiones en la cuál el código, incluyendo el histórico, está replicado en las computadoras de cada desarrollador \cite{wikipedia_2020:Distributed_Version_Control}. Esto permite a cada desarrollador trabajar en el código sin necesidad de estar conectado a un servidor central \cite{juancarlosfernandez_2010}.
Las ventajas que ofrece un sistema distribuido frente a uno centralizado son:
\begin{itemize}
    \item No hay una copia canónica.
    \item Soporta operaciones desconectadas: La mayoría de operaciones comunes como confirmaciones, consultas del historial, diff, y revertir cambios son rápidas, porque no se necesita conexión con el sistema central.
    \item Cada copia de trabajo es una copia de seguridad.
    \item Ramas experimentales: crear y eliminar ramas es rápido y simple.
    \item Fomenta la colaboración entre desarrolladores, permitiendo por ejemplo compartir cambios sin necesidad de confirmarlos en el servidor central.
\end{itemize}

\begin{figure}[H]
    \centering
    \includegraphics[scale=0.16]{sistema-centralizado}
    \caption{Esquema de control de versiones centralizado}
    \label{fig:sistema-centralizado}
\end{figure}


\begin{figure}[H]
    \centering
    \includegraphics[scale=0.16]{sistema-distribuido}
    \caption{Esquema de control de versiones distribuido}
    \label{fig:sistema-distribuido}
\end{figure}


La web de Git tiene un apartado de \href{https://git-scm.com/doc}{documentación} con enlaces a los documentos siguientes:
\begin{itemize}
    \item \href{https://git-scm.com/docs}{Manual de referencia}.
    \item \href{https://github.github.com/training-kit/}{Algunos} \href{https://ndpsoftware.com/git-cheatsheet.html}{cheatsheets}
    \item Libro online \href{https://git-scm.com/book/en/v2}{Pro Git}
    \item Algunos \href{https://git-scm.com/video/what-is-version-control}{videos} \href{https://git-scm.com/video/what-is-git}{básicos} \href{https://git-scm.com/video/get-going}{sobre} \href{https://git-scm.com/video/quick-wins}{Git}.
    \item Una \href{https://git-scm.com/doc/ext}{recopilación de tutoriales}, libros, videos y cursos sobre Git.
\end{itemize}

Aparte de Git, existen otros sistemas de control de versiones distribuidos como por ejemplo Mercurial. Está implementado principalmente haciendo uso del lenguaje de programación Python, pero incluye una implementación binaria de diff escrita en C. Mercurial fue escrito originalmente para funcionar sobre GNU/Linux. Ha sido adaptado para Windows, Mac OS X y la mayoría de otros sistemas tipo Unix. Mercurial es, sobre todo, un programa para la línea de comandos. Todas las operaciones de Mercurial se invocan como opciones dadas a su programa motor, hg (cuyo nombre hace referencia al símbolo químico del mercurio) \cite{wikipedia_2019:Mercurial}.

Según Intland Software, las principales diferencias entre Git y Mercurial son las siguientes \cite{intland_2015:Pros_Cons_Mercurial_Git}:
\begin{itemize}
    \item En ambos sistemas, la historia tiene forma de grafo acíclico. Sin embargo, Mercurial ofrece un histórico lineal simple que puede causar confusión debido a la falta de ifnormación. Git, por el contrario, permite seguir el historial hacia atrás, pero es complicado de hacer.
    \item A menudo se piensa que Git maneja las ramas mejor que Mercurial. La estructura de ramas de Git ayuda a evitar errores en los ``merges'' de código.
    \item Git refuerza la excelencia técnica.
    \item Git es más potente para proyectos grandes.
\end{itemize}

\begin{figure}[H]
    \centering
    \includegraphics[scale=1]{git-mercurial-pros-cons}
    \caption{Pros y contras de Git y Mercurial\cite{intland_2015:Pros_Cons_Mercurial_Git}}
    \label{fig:git-mercurial-pros-cons}
\end{figure}


\section{Ejercicio 2}
\begin{center}
    \parbox{12cm}{\justify\textit{
        Establece tu nombre de usuario y dirección de correo electrónico. Esto es importante porque las confirmaciones de cambios (commits) en Git usan esta información, y es introducida de manera inmutable en los commits que envías. Utiliza tu usuario y correo de la uco, por ejemplo:}}
    \end{center}


\begin{lstlisting}[xleftmargin=.16\textwidth,language=bash]
$ git config --global user.name "i22lojal"
$ git config --global user.email "i22lojal@uco.es"
\end{lstlisting}


\section{Ejercicio 3}
\begin{center}
    \parbox{12cm}{\justify\textit{Vamos a crear un nuevo repositorio desde cero (puedes usar uno tuyo o seguir el breve ejemplo que se describe a continuación). Crea en tu cuenta un nuevo directorio que se llame “reposlycs”. Entra en dicho directorio e inicializa el repositorio (\code{\$git init}). Comprueba la creación del directorio .git. ¿Qué función tiene este directorio?
    }}
\end{center}


\section{Ejercicio 4}
\begin{center}
    \parbox{12cm}{\justify\textit{Crea un directorio que cuelgue del anterior llamado ``include''.
    }}
\end{center}

\section{Ejercicio 5}
\begin{center}
    \parbox{12cm}{\justify\textit{
        En el directorio ``reposlycs'' escribe un fichero llamado “helloworld.c” que contenga el siguiente código:
    }}
\end{center}

\begin{lstlisting}[style=CStyle,xleftmargin=.3\textwidth]
#include <stdio.h>
int main(void)
{
    printf("Hello World!"); 
}\end{lstlisting}
    
\begin{center}
    \parbox{12cm}{\justify\textit{
        y un fichero ``include/myinclude.h'' que contenga:
    }}
\end{center}

\begin{lstlisting}[style=CStyle,xleftmargin=.3\textwidth]
include <stdio.h>
void f()
{
    printf("Hello World \n");
}\end{lstlisting}

\section{Ejercicio 6}
\begin{center}
    \parbox{12cm}{\justify\textit{
        Añade estos ficheros a tu repositorio y comprueba el estado del repositorio con “git status”. Añade los cambios al stage (\code{\$git add ...}). Haz el primer commit con el mensaje ``Initial commit'' y vuelve a comprobar el estado de tu repositorio.
    }}
\end{center}

\section{Ejercicio 7}
\begin{center}
    \parbox{12cm}{\justify\textit{
        Compila y comprueba lo que ocurre con el ejecutable generado. ¿Se podría añadir el ejecutable al staging y luego al commit?. Haz una prueba. ¿Tendría sentido hacer commit de un ejecutable?
    }}
\end{center}

\section{Ejercicio 8}
\begin{center}
    \parbox{12cm}{\justify\textit{
        Abre una rama que se llama ``branchA'' y modifica el fichero de la función f() con lo necesario para que f() quede así:
    }}
\begin{lstlisting}[style=CStyle,xleftmargin=.3\textwidth]
void f()
{
    char c1[100]= "Hello world";
    char c2[100]= ", I am <your name here>";
    printf("%s\n", strcat(c1, c2));
    return 0;
}\end{lstlisting}
\end{center}

\section{Ejercicio 9}
\begin{center}
    \parbox{12cm}{\justify\textit{
        Realiza el commit con los cambios mencionados en la branchA y el mensaje ``Now with strcat()''.
    }}
\end{center}

\section{Ejercicio 10}
\begin{center}
    \parbox{12cm}{\justify\textit{
        Cambia a la rama master (\code{\$git checkout master}). En la rama master añade un comentario al inicio del fichero ``helloworld.c'' con la fecha y tu nombre como autor del fichero. Realiza el commit de este cambio en la rama master. Con el mensaje ``Now with date and author''
    }}
\end{center}

\section{Ejercicio 11}
\begin{center}
    \parbox{12cm}{\justify\textit{
        Desde la rama master, fusiona la rama master con la rama branchA. En este caso no hay ningún conflicto, ambas ramas han modificado distintos archivos. Es un merge sin conflicto.
    }}
\end{center}

\section{Ejercicio 12}
\begin{center}
    \parbox{12cm}{\justify\textit{
        Repite el proceso del punto anterior pero ahora creando un conflicto (merge con conflicto).
    }}
\end{center}

\section{Ejercicio 13}
\begin{center}
    \parbox{12cm}{\justify\textit{
        Sube este proyecto a GitHub.
    }}
\end{center}

\section{Ejercicio 14}
\begin{center}
    \parbox{12cm}{\justify\textit{
        Bájate un proyecto de GitHub de uno de tus compañeros (o de cualquier otra  persona) a tu cuenta local.
    }}
\end{center}