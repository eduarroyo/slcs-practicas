\part{Práctica 4}

\section{Ejercicio 1}
\begin{center}
    \parbox{12cm}{\justify\textit{Uso extendido de GNU/Linux en escritorio y en la industria. Últimos avances.
    Lectura recomendada:
    \begin{itemize}
        \item \href{https://hipertextual.com/2018/12/snaps-instalar-software-linux}{Snaps: ¿El método más simple para instalar apps en Linux?}
    \end{itemize}
    Cuestiones:
    \begin{enumerate}
    \item Porcentaje de uso de los principales SO de escritorio.
    \item ¿Cómo crees que podría mejorarse el uso de GNU/Linux en escritorios?
    \item ¿Qué es snap y snapcraft.io? ¿para qué sirve y qué diferencias aporta al sistema tradicional de instalación de paquetes de GNU/Linux?
    \item Buscar cómo se instala snap en mi distribución en la documentación de snap.
    Instalar y probar snap
    \item ¿Crees que mejora en el sentido de facilitar el uso de GNU/Linux?
    \end{enumerate}}}
\end{center}

\section{Ejercicio 2}
\begin{center}
    \parbox{12cm}{\justify\textit{Free software and GNU/Linux publications:
    \begin{itemize}
    \item LJ. Linux Journal. \url{https://www.linuxjournal.com/}
    \item LM. Linux Magazine. \url{http://www.linux-magazine.com/}
    \item Linux Weekly News. \url{https://lwn.net/}
    \item \url{linux.com}
    \item nixCraft. \url{https://www.cyberciti.biz/}
    \item slashdot y barrapunto
    \item OMG! Ubuntu!. We follow the Ubuntu and GNOME development. \url{https://www.omgubuntu.co.uk/}
    \item Sección “Linux” en: \url{hipertextual.com}
    \item \url{https://ayudalinux.com/}
    \item \url{https://lamiradadelreplicante.com/}
    \end{itemize}
    Cuestiones:
    \begin{enumerate}
    \item Ordena tus preferencias de 1 a 5.
    \item ¿Has encontrado contenido GNU/Linux o de free software en otras publicaciones?¿Cuáles?
    \end{enumerate}}}
\end{center}


\section{Ejercicio 3}
\begin{center}
    \parbox{12cm}{\justify\textit{GNU/Linux en redes sociales:
    \begin{itemize}
    \item \href{https://twitter.com/Linux}{@Linux} 269mil
    \item \href{https://twitter.com/usemoslinux}{@usemoslinux} 24mil
    \item \href{https://twitter.com/MuyLinux}{@MuyLinux} 23mil
    \item \href{https://twitter.com/linuxhispano}{@linuxhispano} 7mil
    \item \href{https://twitter.com/andalinux}{@andalinux} 3mil
    \item \href{https://twitter.com/AulaSL}{@AulaSL} 634
    \end{itemize}
    Best Linux hashtags, o hashtags relacionadas con Linux que más se usan en RRSS:
    \href{https://twitter.com/hashtag/linux}{\#linux}
    \href{https://twitter.com/hashtag/programming}{\#programming}
    \href{https://twitter.com/hashtag/python}{\#python}
    \href{https://twitter.com/hashtag/coding}{\#coding}
    \href{https://twitter.com/hashtag/hacking}{\#hacking}
    \href{https://twitter.com/hashtag/technology}{\#technology}
    \href{https://twitter.com/hashtag/programmer}{\#programmer}
    \href{https://twitter.com/hashtag/technology}{\#technology}
    \href{https://twitter.com/hashtag/java}{\#java}
    \href{https://twitter.com/hashtag/hacker}{\#hacker}
    \href{https://twitter.com/hashtag/computerscience}{\#computerscience}
    \href{https://twitter.com/hashtag/code}{\#code}
    \href{https://twitter.com/hashtag/coder}{\#coder}
    \href{https://twitter.com/hashtag/javascript}{\#javascript}
    \href{https://twitter.com/hashtag/html}{\#html}
    \href{https://twitter.com/hashtag/developer}{\#developer}
    \href{https://twitter.com/hashtag/cybersecurity }{\#cybersecurity }
    \href{https://twitter.com/hashtag/webdeveloper}{\#webdeveloper}
    \href{https://twitter.com/hashtag/css}{\#css}
    \href{https://twitter.com/hashtag/computer}{\#computer}
    \href{https://twitter.com/hashtag/kalilinux}{\#kalilinux}
    \href{https://twitter.com/hashtag/software}{\#software}
    \href{https://twitter.com/hashtag/php}{\#php}
    \href{https://twitter.com/hashtag/webdevelopment}{\#webdevelopment}
    \href{https://twitter.com/hashtag/webdesign}{\#webdesign}
    \href{https://twitter.com/hashtag/programmers}{\#programmers}
    \href{https://twitter.com/hashtag/geek}{\#geek}
    \href{https://twitter.com/hashtag/softwaredeveloper}{\#softwaredeveloper}
    \href{https://twitter.com/hashtag/windows}{\#windows}
    \href{https://twitter.com/hashtag/bhfyp}{\#bhfyp}
    \\
    Cuestiones:
    \begin{enumerate}
    \item Anota las cuentas de Twitter que más te gusten de las anteriores o de otras que tú consultes en orden preferente de 1 a 5.
    \item ¿Has encontrado contenido GNU/Linux o de free software en otras RRSS? ¿Cuáles?
    \end{enumerate}
    Lectura recomendada en el portal Linux Journal sobre su 25 aniversario:
    \begin{itemize}
        \item \href{https://www.linuxjournal.com/content/25-years-later-interview-linus-torvalds}{25 Years Later: Interview with Linus Torvalds}
    \end{itemize}
    }}
\end{center}