\part{Práctica 5}
\begin{center}
    \parbox{12cm}{\justify\textit{Comienza leyendo los siguientes textos:
    \begin{enumerate}
    \item Artículo “Mensajería instantánea libre y responsable”, por Óscar Martín, de Ingeniería sin Fronteras Andalucía.
    \item Acerca de riseup.net y Political Principles
    \item ¿Qué hace especial al correo de riseup.net? 
    \end{enumerate}
    Una vez leído, puedes realizar las siguientes actividades:
    \begin{enumerate}
        \item Instala en tu móvil las aplicaciones de mensajería instantánea sugeridas en el artículo 1. Reflexiona sobre puntos fuertes y débiles de cada una, y en qué se puede basar la popularidad de cada una.
        \item Responde a la pregunta ¿es Telegram totalmente libre?
        \item Investiga sobre el protocolo OTR (Off the record messaging), ¿podrías utilizar este cifrado en alguna de las redes de mensajería instantánea que usas?
        \item Investiga sobre GNU Privacy Guard (GnuPG o GPG). Estudia cómo podrías usarlo en tu correo electrónico (existen implementaciones a nivel de navegador y a nivel de cliente de correo).
    \end{enumerate}
    Otros textos de interés:
    \begin{itemize}
    \item Defensa personal del correo electrónico, infografía de la FSF (Free Software Foundation)
    \item Defensa personal del correo electrónico, guía de la FSF (Free Software Foundation)
    \item Tor and HTTPS, de la EFF (Electronic Frontier Foundation)
    \item NSA Spying on Americans, de la EFF
    \end{itemize}
    }}
\end{center}